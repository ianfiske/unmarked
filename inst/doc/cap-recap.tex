
\documentclass[a4paper]{article}
\usepackage[OT1]{fontenc}
\usepackage{Sweave}
\usepackage{natbib}
\usepackage{fullpage}
\usepackage{amsmath}
\bibliographystyle{plain}

\DefineVerbatimEnvironment{Sinput}{Verbatim} {xleftmargin=2em}
\DefineVerbatimEnvironment{Soutput}{Verbatim}{xleftmargin=2em}
\DefineVerbatimEnvironment{Scode}{Verbatim}{xleftmargin=2em}
\fvset{listparameters={\setlength{\topsep}{0pt}}}
\renewenvironment{Schunk}{\vspace{\topsep}}{\vspace{\topsep}}

%%\VignetteIndexEntry{Capture-recapture}

\title{Capture-recapture models in {\tt unmarked}}
\author{Richard Chandler}


\begin{document}

\maketitle

\abstract{The ``{\tt un}'' in {\tt unmarked} is somewhat misleading
  because the package can be used to analyze data from marked animals. The three
  most common sampling methods that produce suitable data are removal
  sampling, double observer sampling, and capture-recapture
  methods\footnote{Sometimes animals are not actually marked when
    using these methods, but they are treated as though
    they are}. This document focuses on the analysis of capture-recapture
  data using a class of models known as multinomial $N$-mixture
  models \citep{royle_generalized_2004, fiskeChandler_2011}, which can
  be fit using the functions \code{multinomPois} and
  \code{gmultmix}. Capture-recapture models can be fitted with
  constant parameters ($M_0$), time-specific parameters ($M_t$),
  and behavioral responses ($M_b$). In addition, spatial
  variation in abundance or capture probability can also be
  modeled. \pkg{unmarked} has two functions for fitting
  capture-recapture models: \code{multinomPois} and
  \code{gmultmix}. Both allow for user-defined functions to describe
  the capture process, and the latter allows for modeling of temporary
  emigration.
}


\section{Introduction}

Capture-recapture models have a long history in ecology and are used
to estimate population size while controling for variation in capture
probability \citep{williams_etal:2002}. Traditional capture-recapture
models, however, do not
allow one to model spatial variation in abundance---a central
objective of much ecological research. \citet{royle_generalized_2004} proposed a
simple hierarchical model to overcome this limitation. The model
requires that $R$ ``sites'' are surveyed using capture-recapture
methods, and that site-specific abundance ($N_i, i=1,2,...,R$) can be modeled as a discrete
random variable. For example, if assert that that abundance is Poisson
distributed, we can describe the model as
\begin{align}
  N_i \sim \mbox{Poisson}(\lambda) \nonumber \\
  {\bf y_i}|N_i \sim \mbox{Multinomial}(N_i, \pi(p))
  \label{mod}
\end{align}
In the above, $\lambda$ is the expected number of individuals at each
site. ${\bf y_i}$ is a vector containing the number of
individuals with encounter history $k, k=1,2,...K$ at site $i$. The
number of possible encounter histories $K$ depends on the sampling
protocol.
$\pi(p)$ is a function that that converts capture probability ($p$) to
multinomial cell probabilities, \emph{i.e.}, the proportion
of individuals expected to have capture history $k$. The definition of
$\pi(p)$ is also specific to the sampling protocol. For example, in removal
sampling\footnote{a simple type of capture-recapture sampling}, if there are 3
removal passes, the function is
\[
{\bold \pi(p)} = \{p, (1-p)p, (1-p)^2p, (1-p)^3\}.
\]
These 4 multinomial cell probabilities correspond to the 4 possible
encounter histories: $H = (100, 010, 001, 000)$. The probability in
$\pi(p)$ corresponds to the
probability of not being captured. This form of $\pi(p)$ can be
specified in \pkg{unmarked} by setting the argument \code{piFun} to
\code{`removal'} when formatting the data with the
\code{unmarkedFrame} function. Another canned option is for double observer
sampling, \code{piFun=`double'}. At some point in the near future, we
may offer a capture-recapture
\code{piFun}. However, the purpose of this document is to describe how
users can do so in a transparent way so that they can provide
custom functions to suit their needs.

Before we describe how to format data, it is important to mention that,
as with all the abundance models in
\pkg{unmarked}, spatial variation in abundance can be modeled using covariates
with a log-link function
\[
\exp(\lambda_i) = \beta_0 + \beta_1 x_i
\]
where $x_i$ is some site-specific covariate such as habitat type or
elevation say. Note that multiple covariates, continuous or
categorical, can be considered as covariates of $\lambda$. Similarly,
if we had a single covariate of $p$, we could model it using a logit link
\[
logit(p_{ij}) = \alpha_0 + \beta_1 v_{ij}
\]
where $v_{ij}$ is some covariate that may be specific to the site and
observation.


\section{Data}
Earlier we mentioned that the data $\bf y$ must be a $R \times K$
matrix in which each row is the vector of tabulated encounter
histories for animals captured at some site. A site can be defined in
many different ways. For example, a site could be a wetland where an
array of live traps are placed to capture wood turtles. Or a site could be
a point count plot where we collect capture-recapture data on birds by
dividing the survey into time intervals and recording which intervals
each bird is detected in.

The raw data that one collects in a
capture-recapture study is typically of the form shown in Table 1. Each row of the
matrix corresponds to a capture of some individual. Associated with
this capture we need to know the capture occasion, the site at which
it was captured, and perhaps some covariates. Note that if we have
multiple traps at a site, as is often the case, traditional
capture-recapture, and the models considered here, models ignore this
information. Recently developed spatial capture-recapture models,
however, make use of the trap location data to model density and
distance-related heterogeneity in capture probability. SCR models thus
offer important advantages over traditional methods and should be used
when possible if the assumptions are deemed reasonable. See xyz for
more information.

Simple capture-recapture data is shown in Table.


In the absence of individual covariates, these data can be collapsed
and formatted as shown in Table YYY.

SHOW TYPICAL MULTIONOMIAL DATA










\section{Sampling methods}



Although these methods require that individual animals can
be assigned to distinct categories, these methods are often used when individuals are
unmarked. For example, a common study design divides a 10-minute
point count into two or more time intervals, and observers note the
time periods during which each bird was detected. Thus, keeping
track of individuals during a small time frame is similar to
monitoring marked individuals over longer durations.






\section{Capture-recapture Models}



\subsection{Model $M_0$}



\subsection{Model $M_t$}





\subsection{Model $M_b$}




\subsection{Model $M_t$}


\subsection{Model $M_h$}

Cite Royle and Dorazio pg 173 ``In the absence of individual effects
on $p$, the individual encounter histories can be pooled into groups
of unique encounter histories, indexed by $h$, a unique combination of
zeros and ones.''.

{\tt unmarked} does not, however, allow for the modeling of individual
  hetergenity in capture probability although data cou


\section{Analysis in {\tt unmarked}}




\section{Closed population capture-recapture models}




\begin{Schunk}
\begin{Sinput}
> alfl.capRecap <- read.csv(system.file("csv", "alfl.capRecap.csv", package="unmarked"),
                          row.names=1)
> names(alfl.capRecap)
\end{Sinput}
\begin{Soutput}
 [1] "visit1_001" "visit1_010" "visit1_011" "visit1_100" "visit1_101"
 [6] "visit1_110" "visit1_111" "visit2_001" "visit2_010" "visit2_011"
[11] "visit2_100" "visit2_101" "visit2_110" "visit2_111" "visit3_001"
[16] "visit3_010" "visit3_011" "visit3_100" "visit3_101" "visit3_110"
[21] "visit3_111" "struct"     "woody"      "time.1"     "time.2"    
[26] "time.3"     "date.1"     "date.2"     "date.3"    
\end{Soutput}
\end{Schunk}



\begin{Schunk}
\begin{Sinput}
> crPiFun <- function(p) { # p should have 3 columns
     cbind((1-p[,1]) * (1-p[,2]) * p[,3],
           (1-p[,1]) * p[,2] * (1-p[,3]),
           (1-p[,1]) * p[,2] * p[,3],
           p[,1] * (1-p[,2]) * (1-p[,3]),
           p[,1] * (1-p[,2]) * p[,3],
           p[,1] * p[,2] * (1-p[,3]),
           p[,1] * p[,2] * p[,3])
 }
> p <- matrix(0.4, 2, 3)
> crPiFun(p)
\end{Sinput}
\begin{Soutput}
      [,1]  [,2]  [,3]  [,4]  [,5]  [,6]  [,7]
[1,] 0.144 0.144 0.096 0.144 0.096 0.096 0.064
[2,] 0.144 0.144 0.096 0.144 0.096 0.096 0.064
\end{Soutput}
\begin{Sinput}
> rowSums(crPiFun(p))
\end{Sinput}
\begin{Soutput}
[1] 0.784 0.784
\end{Soutput}
\end{Schunk}


obsToY needs to be a matrix with
the number of rows equal to the number of columns for some obsCov, and
the number columns equal to the number of columns in y
If obsToY[i,j] is 1, then a missing value in obsCov translates to
a missing value in y

\begin{Schunk}
\begin{Sinput}
> o2y <- matrix(1, 3, 7)
\end{Sinput}
\end{Schunk}



\begin{Schunk}
\begin{Sinput}
> visitMat <- matrix(c('V1','V2','V3'), 50, 3, byrow=TRUE)
> visitMat[1,2] <- NA
> visitMat[2,] <- NA
> head(visitMat)
\end{Sinput}
\begin{Soutput}
     [,1] [,2] [,3]
[1,] "V1" NA   "V3"
[2,] NA   NA   NA  
[3,] "V1" "V2" "V3"
[4,] "V1" "V2" "V3"
[5,] "V1" "V2" "V3"
[6,] "V1" "V2" "V3"
\end{Soutput}
\begin{Sinput}
> library(unmarked)
> umf.cr1 <- unmarkedFrameMPois(y=alfl.capRecap[,1:7],
                         siteCovs=alfl.capRecap[,c("woody", "struct")],
                         obsCovs=list(visit=visitMat),
                         obsToY=o2y, piFun="crPiFun")
> (M0 <- multinomPois(~1 ~1, umf.cr1))
\end{Sinput}
\begin{Soutput}
Call:
multinomPois(formula = ~1 ~ 1, data = umf.cr1)

Abundance:
 Estimate    SE      z P(>|z|)
  0.00714 0.141 0.0504    0.96

Detection:
 Estimate    SE    z  P(>|z|)
     1.43 0.216 6.63 3.41e-11

AIC: 258.2382 
\end{Soutput}
\end{Schunk}




\begin{Schunk}
\begin{Sinput}
> (Mt <- multinomPois(~visit ~1, umf.cr1))
\end{Sinput}
\begin{Soutput}
Call:
multinomPois(formula = ~visit ~ 1, data = umf.cr1)

Abundance:
 Estimate    SE       z P(>|z|)
  -0.0137 0.146 -0.0938   0.925

Detection:
            Estimate    SE      z  P(>|z|)
(Intercept)    1.402 0.371  3.779 0.000157
visitV2        0.292 0.543  0.538 0.590599
visitV3       -0.248 0.499 -0.496 0.619611

AIC: 250.449 
\end{Soutput}
\end{Schunk}



\section{Capture-recapture models allowing for temporary emigration}



\bibliography{unmarked}



\end{document}
