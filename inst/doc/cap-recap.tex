
\documentclass[a4paper]{article}
\usepackage[OT1]{fontenc}
\usepackage{Sweave}
\usepackage{natbib}
\usepackage{fullpage}
\usepackage{amsmath}
\bibliographystyle{plain}

\DefineVerbatimEnvironment{Sinput}{Verbatim} {xleftmargin=2em}
\DefineVerbatimEnvironment{Soutput}{Verbatim}{xleftmargin=2em}
\DefineVerbatimEnvironment{Scode}{Verbatim}{xleftmargin=2em}
\fvset{listparameters={\setlength{\topsep}{0pt}}}
\renewenvironment{Schunk}{\vspace{\topsep}}{\vspace{\topsep}}

%%\VignetteIndexEntry{Capture-recapture}

\title{Capture-recapture models in {\tt unmarked}}
\author{Richard Chandler}


\begin{document}

\maketitle

\abstract{The ``{\tt un}'' in {\tt unmarked} is somewhat misleading
  because the package can be used to analyze data from marked animals. The three
  most common sampling methods that produce suitable data are removal
  sampling, double observer sampling, and capture-recapture
  methods\footnote{Sometimes animals are not actually marked when
    using these methods, but they are treated as though
    they are}. This document focuses on the analysis of capture-recapture
  data using a class of models known as multinomial $N$-mixture
  models \citep{royle_generalized_2004, fiskeChandler_2011}, which
  assume that capture-recapture data have been collected at a
  collection of sample locations (``sites''). Capture-recapture models
  can be fitted with
  constant parameters ($M_0$), time-specific parameters ($M_t$),
  and behavioral responses ($M_b$). In addition, spatial
  variation in abundance or capture probability can also be
  modeled using site-specific covariates. \pkg{unmarked} has two
  functions for fitting
  capture-recapture models: \code{multinomPois} and
  \code{gmultmix}. Both allow for user-defined functions to describe
  the capture process, and the latter allows for modeling of temporary
  emigration.
}


\section{Introduction}

Capture-recapture models have a long history in ecology and are used
to estimate population size while controling for variation in capture
probability \citep{williams_etal:2002}. Traditional capture-recapture
models, however, do not
allow one to model spatial variation in abundance---a central
objective of much ecological research. \citet{royle_generalized_2004} proposed a
simple hierarchical model to overcome this limitation. The model
requires that $R$ ``sites'' are surveyed using capture-recapture
methods, and that site-specific abundance ($N_i; i=1,2,...,R$) can be modeled as a discrete
random variable. For example, if abundance is Poisson
distributed, we can describe the model as
\begin{gather}
  N_i \sim \mbox{Poisson}(\lambda) \nonumber \\
  {\bf y_i}|N_i \sim \mbox{Multinomial}(N_i, \pi(p))
  \label{mod}
\end{gather}
In the above, $\lambda$ is the expected number of individuals at each
site. ${\bf y_i}$ is a vector containing the number of
individuals with encounter history $k; k=1,2,...K$ at site $i$. The
number of observable encounter histories $K$ depends on the sampling
protocol. For a capture-recapture study with 2 time periods, $K$
equals 3 because the possibilities are $(11, 10, 01)$. In Equation~\ref{mod},
$\pi(p)$ is a function that that converts capture probability ($p$) to
multinomial cell probabilities, \emph{i.e.}, the proportion
of individuals expected to have capture history $k$. The definition of
$\pi(p)$ is also specific to the sampling protocol. For example, the
cell probabilities corresponding to the capture histories listed above
are
\[
{\bold \pi(p)} = \{p^2, p(1-p), (1-p)p, (1-p)^2\}.
\]
Note that the last multinomial cell probability corresponds to the
probability of not being captured.

Traditional capture-recapture methods were developed for sampling
situations in which a single site was surveyed using trap
arrays, and the objective was to estimate the number of individuals
exposed to sampling $N$. Since there was not variation in abundance to
model, emphasis was placed on
modeling variation in capture probability ($p$). [Mention standard models]
Often, however,
ecologists collect data at a collection of sites, and interest lies in
describing variation in abundance among those sites. For instance,
capture-recapture arrays might be established in replicates sites in 2
different habitat types to make inferences about differences in
abundance among the two habitats. as with all the abundance models in
\pkg{unmarked}, spatial variation in abundance can be modeled using covariates
with a log-link function
\[
\exp(\lambda_i) = \beta_0 + \beta_1 x_i
\]
where $x_i$ is some site-specific covariate such as habitat type or
elevation. A more general form is written as
\[
\exp(\lambda_i) = {\bold X_i' \beta}
\]
where ${\bold X}$ is a design matrix and ${\bold \beta}$ is a vector
of coefficients, possibly including an intercept.
Capture probability can be modeled in the same way using the logit
instead of the log link. For instance, we could have
\[
logit(p_{ij}) = \alpha_0 + \alpha_1 v_{ij}
\]
where $v_{ij}$ is some covariate specific to the site and
capture occasion.


\section{Data}
Earlier we mentioned that the data $\bf y$ must be a $R \times K$
matrix in which each row is the vector of tabulated encounter
histories for animals captured at some site. Capture-recapture data,
however, is typically recorded in the format shown in Table\ref

\begin{table}[h]
  \centering
  \caption{Capture-recapture data for 3 individuals. There were 3
    trapping occasions}
  \begin{tabular}{lcc}
    \hline
    Animal ID   & Site  & Capture history \\
    \hline
    GB        & A     & 101 \\
    RO        & A     & 111 \\
    GY        & B     & 100 \\
    \hline
  \end{tabular}
  \label{tab:raw}
\end{table}

In the absence of individual covariates, the data shown in Table~\ref{tab:raw} can be collapsed
and formatted as shown in Table~\ref{tab:format}.

\begin{table}[h]
  \centering
  \caption{Capture-recapture data in the format required by \pkg{unmarked}}
  \begin{tabular}{lccccccc}
    \hline
    Site  & 100 & 010 & 001 & 110 & 011 & 101 & 111 \\
    \hline
    A     & 0   & 2   & 3   & 0   & 1   & 1   & 0   \\
    B     & 0   & 2   & 3   & 0   & 1   & 1   & 0   \\
    \hline
  \end{tabular}
\end{table}







\section{Analysis in \pkg{unmarked}}



\subsection{Model $M_0$}



\subsection{Model $M_t$}





\subsection{Model $M_b$}




\subsection{Model $M_h$}

Cite Royle and Dorazio pg 173 ``In the absence of individual effects
on $p$, the individual encounter histories can be pooled into groups
of unique encounter histories, indexed by $h$, a unique combination of
zeros and ones.''.

{\tt unmarked} does not, however, allow for the modeling of individual
  hetergenity in capture probability although data cou


\section{Analysis in {\tt unmarked}}




\section{Closed population capture-recapture models}




\begin{Schunk}
\begin{Sinput}
> alfl.capRecap <- read.csv(system.file("csv", "alfl.capRecap.csv", 
     package = "unmarked"), row.names = 1)
> names(alfl.capRecap)
\end{Sinput}
\begin{Soutput}
 [1] "visit1_001" "visit1_010" "visit1_011" "visit1_100" "visit1_101"
 [6] "visit1_110" "visit1_111" "visit2_001" "visit2_010" "visit2_011"
[11] "visit2_100" "visit2_101" "visit2_110" "visit2_111" "visit3_001"
[16] "visit3_010" "visit3_011" "visit3_100" "visit3_101" "visit3_110"
[21] "visit3_111" "struct"     "woody"      "time.1"     "time.2"    
[26] "time.3"     "date.1"     "date.2"     "date.3"    
\end{Soutput}
\end{Schunk}



\begin{Schunk}
\begin{Sinput}
> crPiFun <- function(p) {
     cbind((1 - p[, 1]) * (1 - p[, 2]) * p[, 3], (1 - 
         p[, 1]) * p[, 2] * (1 - p[, 3]), (1 - p[, 1]) * 
         p[, 2] * p[, 3], p[, 1] * (1 - p[, 2]) * (1 - 
         p[, 3]), p[, 1] * (1 - p[, 2]) * p[, 3], p[, 
         1] * p[, 2] * (1 - p[, 3]), p[, 1] * p[, 2] * 
         p[, 3])
 }
> p <- matrix(0.4, 2, 3)
> crPiFun(p)
\end{Sinput}
\begin{Soutput}
      [,1]  [,2]  [,3]  [,4]  [,5]  [,6]  [,7]
[1,] 0.144 0.144 0.096 0.144 0.096 0.096 0.064
[2,] 0.144 0.144 0.096 0.144 0.096 0.096 0.064
\end{Soutput}
\begin{Sinput}
> rowSums(crPiFun(p))
\end{Sinput}
\begin{Soutput}
[1] 0.784 0.784
\end{Soutput}
\end{Schunk}


obsToY needs to be a matrix with
the number of rows equal to the number of columns for some obsCov, and
the number columns equal to the number of columns in y
If obsToY[i,j] is 1, then a missing value in obsCov translates to
a missing value in y

\begin{Schunk}
\begin{Sinput}
> o2y <- matrix(1, 3, 7)
\end{Sinput}
\end{Schunk}



\begin{Schunk}
\begin{Sinput}
> visitMat <- matrix(c("V1", "V2", "V3"), 50, 3, byrow = TRUE)
> visitMat[1, 2] <- NA
> visitMat[2, ] <- NA
> head(visitMat)
\end{Sinput}
\begin{Soutput}
     [,1] [,2] [,3]
[1,] "V1" NA   "V3"
[2,] NA   NA   NA  
[3,] "V1" "V2" "V3"
[4,] "V1" "V2" "V3"
[5,] "V1" "V2" "V3"
[6,] "V1" "V2" "V3"
\end{Soutput}
\begin{Sinput}
> library(unmarked)
> umf.cr1 <- unmarkedFrameMPois(y = alfl.capRecap[, 1:7], 
     siteCovs = alfl.capRecap[, c("woody", "struct")], 
     obsCovs = list(visit = visitMat), obsToY = o2y, piFun = "crPiFun")
> (M0 <- multinomPois(~1 ~ 1, umf.cr1))
\end{Sinput}
\begin{Soutput}
Call:
multinomPois(formula = ~1 ~ 1, data = umf.cr1)

Abundance:
 Estimate    SE      z P(>|z|)
  0.00714 0.141 0.0504    0.96

Detection:
 Estimate    SE    z  P(>|z|)
     1.43 0.216 6.63 3.41e-11

AIC: 258.2382 
\end{Soutput}
\end{Schunk}




\begin{Schunk}
\begin{Sinput}
> (Mt <- multinomPois(~visit ~ 1, umf.cr1))
\end{Sinput}
\begin{Soutput}
Call:
multinomPois(formula = ~visit ~ 1, data = umf.cr1)

Abundance:
 Estimate    SE       z P(>|z|)
  -0.0137 0.146 -0.0938   0.925

Detection:
            Estimate    SE      z  P(>|z|)
(Intercept)    1.402 0.371  3.779 0.000157
visitV2        0.292 0.543  0.538 0.590599
visitV3       -0.248 0.499 -0.496 0.619611

AIC: 250.449 
\end{Soutput}
\end{Schunk}



\section{Capture-recapture models allowing for temporary emigration}



\section{Spatially-explicit Capture-recapture Models}

Recently developed spatial capture-recapture models,
however, make use of the trap location data to model density and
distance-related heterogeneity in capture probability. SCR models thus
offer important advantages over traditional methods and should be used
when possible if the assumptions are deemed reasonable. See xyz for
more information.



\bibliography{unmarked}



\end{document}
